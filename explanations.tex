\documentclass{article} \usepackage{amsmath} \usepackage[utf8]{inputenc}

\title{rsa} \author{Philippe Carphin} \date{April 2020}

\setlength{\parskip}{2mm} \setlength{\parindent}{0em}
\begin{document}

\maketitle

\section{Introduction}
RSA encryption is mostly based on these three things:
\begin{enumerate}


\item Factoring a large number is way hard
\item Euler's totient function $\phi(n)$, which says how many numbers between
  $1$ and $n-1$ do not share factors with $n$, is
\begin{itemize}
    \item Hard to calculate if you do not know the factors of $n$
    \item Easy to calculate if you know the factors of $n$.
\end{itemize}
\item $a^{k*\phi(n)} \equiv 1 \mod{n}$ (this is known as Fermat's little
  theorem).

\end{enumerate}
Also sidenote on modular arithmetic in math versus in computer science. In math,
we say that $11 \equiv 4 \mod{7}$ meaning that we consider them the same
$\mod{7}$ and we will be considering everything $\mod{n}$. We will consider two
numbers to be equivalent $\mod{n}$ if they differ by a multiple of $n$. The
following
\begin{align}
    (a + kn) \cdot (b + ln) &= ab + aln + bkn + kln^2 \\ &= ab + (al + bk +
  kln)n \\ (a + kn) + (b + ln) &= a + b + (k+l)n
\end{align}
shows that $ab \equiv (a+kn)\dot(b+ln) \mod{n}$ This means that in modular
arithmetic we can replace any of the numbers with another one that is equivalent
modulo $n$ and the result will be equivalent modulo $n$.
\begin{align}
    15 \cdot 71 \equiv 1 \cdot 1 \mod{7}
\end{align}

For example
\begin{align}
    15 \cdot 71 &= 1065
\end{align}
but if we were just interested in the result modulo $7$ we need only note that
$15 \equiv 1 \mod{7}$ and $71 \equiv 1 \mod{7}$. So the result $\mod{7}$ is just
$1$.

\section{Initial setup and idea}
 The idea is to take $p,q$, two large primes then calculate \[n = pq.\]

 For a prime number $p$, no other numbers than $p$ itself are factors of $p$, so
 $\phi(p) = p-1$. You'll have to trust me that $\phi(mn) = \phi(m)\phi(n)$. We
 can therefore calculate $\phi(n) = \phi(p)\phi(q) = (p-1)(q-1)$.
 

 All we have left to do is to find two things that when multiplied together make
 $k*\phi(n) + 1$. Then we will have
\begin{align}
    a^{k\phi(n) + 1} &\equiv a^{k\phi(n)}\cdot a \mod{n} \\ &\equiv 1 \cdot a
    \mod{n} \\ &\equiv a \mod{n}
\end{align}

We will find $e,d$ such that $e\cdot d = k\phi(n) + 1$. We will publish $e,n$ so
that someone may take their message $m$, encrypt it by calculating $m^e
\mathrm{mod}\ n$. They will send us this $m^e$ and because we know $d$, we will
be able to get $m$ back:

\begin{align}
    (m^e)^d &= m^{ed} \\ &= m^{k\phi(n) + 1} \\ &\equiv m\ \mod{n}
\end{align}

\section{Final setup}

 We start by trying to find $e$ such that $\gcd(e,\phi(n)) = 1$.
 
In my implementation, I just try different values for $e$ and check whether
$\gcd(e, \phi(n)) = 1$.

It is a theorem in mathematics that for any $a, b$ there exist $x, y$ such that
\[xa + yb = \gcd(a,b).\]

In our case, that means that if we have found an $e$ that has $\gcd(e, \phi(n))
= 1$, then there exists $d,k$ such that
\[de * k\phi(n) = 1.\]

\section{Using the algorithm}

We start with $p,q$ two large prime numbers. We calculate their product $n$,
then we calculate $\phi(n) = (p-1)(q-1)$ and find $e,d,k$ such that $ed +
k\phi(n) = 1$ (we can see this as meaning that $ed$ is one above a multiple of
$\phi(n)$, which is the important thing for us).


Let $m < n$ be our message.

Next $m^e \mod{n}$ is the encrypted message, lets call it $w$ (upside down $m$).

 To decrypt the encrypted message w, and we raise it to the power 'd'
\begin{align}
 w^d &\equiv (m^e)^d \mod{n} \\ &\equiv m^(ed) \mod{n} \\ &\equiv m^(y*phi + 1)
 \mod{n} \\ &\equiv m^(y*phi) * m \mod{n} \\ &\equiv 1 * m \mod{n} \\ &\equiv m
 \mod{n}
\end{align}
\end{document}
